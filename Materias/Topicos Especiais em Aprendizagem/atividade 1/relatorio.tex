\documentclass[a4paper, 11pt]{article}
\usepackage[brazilian]{babel}
\usepackage[utf8]{inputenc}
\usepackage[T1]{fontenc}
\usepackage{amsmath}

\usepackage{comment} % enables the use of multi-line comments (\ifx \fi) 
\usepackage{lipsum} %This package just generates Lorem Ipsum filler text. 
\usepackage{fullpage} % changes the margin

% Default fixed font does not support bold face
\DeclareFixedFont{\ttb}{T1}{txtt}{bx}{n}{12} % for bold
\DeclareFixedFont{\ttm}{T1}{txtt}{m}{n}{12}  % for normal

% Custom colors
\usepackage{color}
\definecolor{deepblue}{rgb}{0,0,0.5}
\definecolor{deepred}{rgb}{0.6,0,0}
\definecolor{deepgreen}{rgb}{0,0.5,0}

\usepackage{listings}

% Python style for highlighting
\newcommand\pythonstyle{\lstset{
language=Python,
basicstyle=\ttm,
otherkeywords={self},             % Add keywords here
keywordstyle=\ttb\color{deepblue},
emph={MyClass,__init__},          % Custom highlighting
emphstyle=\ttb\color{deepred},    % Custom highlighting style
stringstyle=\color{deepgreen},
frame=tb,                         % Any extra options here
showstringspaces=false            % 
}}

% Python environment
\lstnewenvironment{python}[1][]
{
\pythonstyle
\lstset{#1}
}
{}

% Python for external files
\newcommand\pythonexternal[2][]{{
\pythonstyle
\lstinputlisting[#1]{#2}}}

% Python for inline
\newcommand\pythoninline[1]{{\pythonstyle\lstinline!#1!}}


\begin{document}
%Header-Make sure you update this information!!!!
\noindent
\large\textbf{Atividade 1} \hfill \textbf{Yan A. S. Duarte} \\
\normalsize PEL 208 \hfill Tópicos Especiais em Aprendizagem \\
Prof. Reinaldo A. C. Bianchi \hfill Entrega: 11/10/2017 \\


\section*{Introdução}

Esse relatório tem como objetivo detalhar a teoria, a implementação, os resultados e a conclusão da primeira atividade do curso. A propósta do exercício é implementar o método dos mínimos quadrados, tanto o básico (linear), quanto o quadrático.

\section*{Teoria}
\subsection*{Matriz Inversa}
Para a execução do método dos mínimos quadrados é necessário obter a matriz inversa.
Uma matriz é inversa de outra quando o produto de ambas resulta em uma matriz identidade, ou seja:
$$ A.A^{-1} = A^{-1}.A = I $$
Para que isso aconteça, a matriz $ A $ deve ser uma matriz quadrada (mesmo número de linhas e colunas).

Podemos encontrar a matriz inversa $ A^{-1} $ com a seguinte fórmula:
$$ A^{-1} = \frac{1}{det(A)} adj(A) $$
onde:
\begin{itemize}
\item $ det(A) $: é a determinante da matriz $ A $ que representa o valor numérico da matriz;
\item $ adj(A) $: é a matriz adjunta da matriz $ A $ que representa a transposta de sua matriz dos cofatores.
\end{itemize}

Para definirmos o que é um cofator, é necessário, a princípio, definirmos o que é o menor principal ou menor complementar associado a um elemento qualquer de uma matriz quadrada.
Essa definição se dá pela determinante de uma matriz $ D $ obtida a partir da eliminação de uma linha i e uma coluna j da matriz $ A $.

Sendo assim, podemos assumir que um cofator $ \tilde{a}_{ij} $ associado a um elemento $ a_{ij} $ é definido por:
$$ \tilde{a}_{ij} = (-1)^{i+j} D_ {ij} $$

Logo, a matriz de cofatores $ C $ é formada por todos os cofatores da matriz original $ A $.

\subsection*{Mínimos Quadrados}
O método dos mínimos quadrado é uma técninca de otimização matemática que possui como objetivo encontrar o melhor ajuste para um conjunto de dados. Seu funcionamento se dá através da minimização da somatória da diferença entre o valor estimado e do dado observado elevado ao quadrado:

$$ \min{ \sum(y_i - \hat{y}_i)^2 } $$

Podemos expressar sua notação matricial da seguinte forma:

$$ \beta = (X^T X)^{-1} X^Ty $$

\section*{Implementação}
Para a elaboração do exercício foi utilizada a linguagem de programação Python.
Foram criadas funções separadas para calcular cada etapa da fórmula dos mínimos quadrados que serão descritas a seguir.

\subsection*{check\_tamanho\_matriz}
Função para exibir retorno de erro caso a matriz de entrada tenha dimensão diferente de 2x2 ou 3x3.

\begin{python}[language=Python]
def check_tamanho_matriz(matriz):
    lin, col = matriz.shape
    
    if (lin!=col): 
        return { 
            'success': False, 
            'message': 'Matriz deve ser quadrada'
        }
    
    if (lin < 2 or lin > 3):
        return {
            'success': False, 
            'message': 'Apenas matrizes de tamanho 2x2 '+
                       'ou 3x3 sao aceitas'
        }
    
    return{
        'success': True, 
        'message': 'Tamanho da matriz: {}x{}'.format(lin, col)
    }
\end{python}

\newpage
\subsection*{calc\_determinante}
Função que calcula o determinante de uma matriz de dimensão 2x2 ou 3x3.

\begin{python}[language=Python]
def calc_determinante(matriz):
    lin, col = matriz.shape
    tamanho = check_tamanho_matriz(matriz)
    
    if not tamanho.get('success'):
        return {
            'success': False
            , 'message': tamanho.get('message')
            , 'result': None
        }
    
    if (lin == 2):
        return {
            'success': True
            , 'message': 'Determinante obtida com sucesso.'
            , 'result': ( matriz[0][0] * matriz[1][1] ) - 
                        ( matriz[0][1] * matriz[1][0] )
        }
    
    if (lin == 3):
        det = 0
        for i in range(lin):
            det = det + (
                matriz[0][i] * 
                (matriz[1][(i+1)%3] * matriz[2][(i+2)%3] - 
                 matriz[1][(i+2)%3] * matriz[2][(i+1)%3])
            );
        return {
            'success': True
            , 'message': 'Determinante obtida com sucesso.'
            , 'result': det
        }
\end{python}

\newpage
\subsection*{calc\_adjunta}
Função que calcula a matriz adjunta de uma matriz de dimensão 2x2 ou 3x3.

\begin{python}[language=Python]
def calc_adjunta(matriz):
    lin, col = matriz.shape
    tamanho = check_tamanho_matriz(matriz)
    
    if not tamanho.get('success'):
        return {
            'success': False
            , 'message': tamanho.get('message')
            , 'result': None
        }
    
    if (lin == 2):
        adj = np.array([
                [matriz[1][1], -matriz[1][0]],
                [-matriz[0][1], matriz[0][0]]
            ])
    
    if (lin == 3):
        adj = np.zeros((3,3))
        
        for i in range(lin):
            for j in range(col):
                adj[i][j] = calc_determinante(np.array([
                    [matriz[(i+1)%3][(j+1)%3], 
                        matriz[(i+1)%3][(j+2)%3]],
                    [matriz[(i+2)%3][(j+1)%3], 
                        matriz[(i+2)%3][(j+2)%3]]
                ])).get('result', None) 
    
    return {
            'success': True
            , 'message': 'Adjunta obtida com sucesso.'
            , 'result': np.transpose(adj)
        }
\end{python}


\newpage
\subsection*{calc\_inversa}
Função que calcula a matriz inversa de uma matriz de dimensão 2x2 ou 3x3.

\begin{python}[language=Python]
def calc_inversa(matriz):
    lin, col = matriz.shape
    tamanho = check_tamanho_matriz(matriz)
    
    if not tamanho.get('success'):
        return {
            'success': False
            , 'message': tamanho.get('message')
            , 'result': None
        }
    
    return {
        'success': True
        , 'message': 'Inversa obtida com sucesso'
        , 'result': 
            (1/calc_determinante(matriz).get('result')) * 
            calc_adjunta(matriz).get('result')
    }
\end{python}

\newpage
\subsection*{calc\_min\_quadrado}
Função que calcula a matriz inversa de uma matriz de dimensão 2x2 ou 3x3.

\begin{python}[language=Python]
def calc_min_quadrado(X, y, quadratico=False):
    lin, col = X.shape
    
    if (quadratico):
        X = np.array([np.append(i, i[col-1]*i[col-1]) for i in X])
    
    inv = calc_inversa(np.dot(np.transpose(X), X))
    
    if not inv.get('success'):
        return {
            'success': False
            , 'message': 'Nao foi possivel calcular o minimo '+
                         'quadrado.\n {}'.format(inv.get('message'))
            , 'result': None
        }
    
    return {
        'success': True
        , 'message': 'Minimo quadrado {} calculado com '+
                     'sucesso.'.format(
                         '(quadratico)' if quadratico else ''
                     )
        , 'result': np.dot(
                        inv.get('result'),
                        np.dot(np.transpose(X), y)
                    )
    }
\end{python}

\subsection*{Testes}
Para testar o funcionamento das funções foram utilizados os seguintes conjuntos de dados:
\begin{itemize}
\item Height x Shoe size;
\item Boiling point at the Alps;
\item Books x Grades;
\item US Census.
\end{itemize}


\newpage
\section*{Resultados}
\subsection*{Height x Shoe size}

Mínimo quadrado linear.
$$ \beta =
\begin{bmatrix}
-25.65123789 \\
  0.51453175
\end{bmatrix}
$$

Mínimo quadrado quadratico.
$$ \beta =
\begin{bmatrix}
-4.21942130e+06 \\
-5.40741892e+00\\
 4.28048409e-02
\end{bmatrix}
$$

\subsection*{Boiling point at the Alps}
Mínimo quadrado linear.
$$ \beta =
\begin{bmatrix}
-81.06372713 \\
  0.5228924
\end{bmatrix}
$$

Mínimo quadrado quadratico.
$$ \beta =
\begin{bmatrix}
 3.88292988e+01 \\
-6.54770851e-01\\
 2.88971271e-03
\end{bmatrix}
$$


\subsection*{Books x Grades}
Mínimo quadrado linear.
$$ \beta =
\begin{bmatrix}
37.3791852 \\
 4.03689261\\
 1.28347727
\end{bmatrix}
$$

Não foi possível obter o mínimo quadrado quadrático desse conjunto de dados tendo em vista que o mesmo possúi duas variáreis de entrada ($x_1$ e $x_2$). Isso faz com que, caso eleve uma de suas variáreis ao quadrado, a matriz resultante da multiplicação $ (X^T X) $ resultará em uma matriz de dimensão maior que 3x3, impossibilitando o algorítmo de funcionar.

\subsection*{US Census}
Mínimo quadrado linear.
$$ \beta =
\begin{bmatrix}
-3.78394559e+03 \\
2.02530273e+00
\end{bmatrix}
$$

Mínimo quadrado quadratico.
$$ \beta =
\begin{bmatrix}
-2.44524585e+09 \\
2.75670665e+09\\
6.95532552e+05
\end{bmatrix}
$$



% to comment sections out, use the command \ifx and \fi. Use this technique when writing your pre lab. For example, to comment something out I would do:
%  \ifx
%	\begin{itemize}
%		\item item1
%		\item item2
%	\end{itemize}	
%  \fi

\section*{Conclusão}
\lipsum[5]

%\begin{thebibliography}{9}
%\bibitem{Robotics} Fred G. Martin \emph{Robotics Explorations: A Hands-On Introduction to Engineering}. New Jersey: Prentice Hall.
%\bibitem{Flueck}  Flueck, Alexander J. 2005. \emph{ECE 100}[online]. Chicago: Illinois Institute of Technology, Electrical and Computer Engineering Department, 2005 [cited 30
%August 2005]. Available from World Wide Web: (http://www.ece.iit.edu/~flueck/ece100).
%\end{thebibliography}

\end{document}
